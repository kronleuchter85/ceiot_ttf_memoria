%----------------------------------------------------------------------------------------
% Preambulo y Configuración
%----------------------------------------------------------------------------------------

\documentclass[
    11pt,
    spanish,
    singlespacing,
    parskip,
    headsepline,
    bookmarks=true,
    unicode=true,
    pdftoolbar=true,
    pdfmenubar=true,
    pdffitwindow=false,
    colorlinks=true,
    linkcolor=blue,
    citecolor=blue,
    urlcolor=blue
]{MastersDoctoralThesis}

\usepackage[utf8]{inputenc} % Codificación de entrada UTF-8
\usepackage[T1]{fontenc}    % Codificación de salida para caracteres especiales
\usepackage{graphicx}       % Manejo de gráficos
\usepackage{eso-pic}        % Permite agregar fondos
\usepackage{hyperref}       % Manejo de hipervínculos y marcadores

% Redefinición de caracteres problemáticos en marcadores
\hypersetup{
    pdftitle={Título del Documento},
    pdfauthor={Autor del Documento},
    pdfkeywords={Sistemas Embebidos, Internet de las Cosas, Inteligencia Artificial},
    pdfstartview={FitH},
    unicode=true,
    colorlinks=true,
    linkcolor=blue,
    citecolor=blue,
    urlcolor=blue
}

\pdfstringdefDisableCommands{%
  \def\texttt#1{#1}%
  \def\textbf#1{#1}%
  \def\textit#1{#1}%
  \def\"{\"}%
  \def\~{~}%
  \def\'{'}%
  \def\^{}%
  \def\textunderscore{\_} % Manejo del subrayado en marcadores
}


% Definir comandos requeridos por la clase
\newcommand{\degreename}{Maestría en Ciencias} % Cambia según tu título
\newcommand{\univname}{Universidad Nacional de Ejemplo} % Cambia según tu universidad
\newcommand{\keywordnames}{Palabras clave:}
%----------------------------------------------------------------------------------------
% Documento Principal
%----------------------------------------------------------------------------------------

\begin{document}

% Configuración de la portada
\posgrado{Carrera / Maestría}
\keywords{Sistemas Embebidos, Internet de las Cosas, Inteligencia Artificial}

% Incluir la portada desde un archivo separado
\include{portada}

% Configuración del contenido preliminar
\frontmatter % Usar numeración romana para las páginas preliminares
\pagestyle{plain} % Estilo de encabezado simple

%----------------------------------------------------------------------------------------
% Resumen
%----------------------------------------------------------------------------------------

\begin{abstract}
\addchaptertocentry{\abstractname} % Agregar resumen al índice
El resumen debe escribirse en uno o dos párrafos. Debe ser breve y conciso, sin ningún elemento de formato en el texto como itálicas o negritas. Tampoco se deben usar siglas ni acrónimos que no resulten obvios para un lector promedio de la memoria, ni referencias bibliográficas o notas al pie de página. No debe faltar qué es lo que se hizo/logró, qué importancia/valor tiene el proyecto/resultado, qué va a encontrar el lector en la memoria y qué contenidos de la especialización/maestría se aplicaron en el proyecto.
\end{abstract}

%----------------------------------------------------------------------------------------
% Agradecimientos
%----------------------------------------------------------------------------------------

\begin{acknowledgements}
\vspace{1.5cm}
Esta sección es para agradecimientos personales y es totalmente \textbf{OPCIONAL}.
\end{acknowledgements}

%----------------------------------------------------------------------------------------
% Índice
%----------------------------------------------------------------------------------------

\tableofcontents
\listoffigures
\listoftables

%----------------------------------------------------------------------------------------
% Dedicatoria
%----------------------------------------------------------------------------------------

\dedicatory{\textbf{Dedicado a... [OPCIONAL]}}

%----------------------------------------------------------------------------------------
% Capítulos
%----------------------------------------------------------------------------------------

\mainmatter % Iniciar numeración numérica para el contenido principal
\pagestyle{thesis} % Estilo de encabezado de tesis

% Incluir capítulos desde archivos separados
% Chapter 1

\chapter{Introducción general} % Main chapter title

\label{Chapter1} % For referencing the chapter elsewhere, use \ref{Chapter1} 
\label{IntroGeneral}

Esta sección presenta la motivación, alcance, objetivos y requerimientos del producto en el marco del estado del arte y su importancia en la industria.

%----------------------------------------------------------------------------------------

% Define some commands to keep the formatting separated from the content 
\newcommand{\keyword}[1]{\textbf{#1}}
\newcommand{\tabhead}[1]{\textbf{#1}}
\newcommand{\code}[1]{\texttt{#1}}
\newcommand{\file}[1]{\texttt{\bfseries#1}}
\newcommand{\option}[1]{\texttt{\itshape#1}}
\newcommand{\grados}{$^{\circ}$}

%----------------------------------------------------------------------------------------

%\section{Introducción}

%----------------------------------------------------------------------------------------
\section{Estado del arte}

Las soluciones IoT (\textit{Internet of Things} o Internet de las Cosas) se basan en la conexión de dispositivos físicos con sistemas software para recopilar, transmitir y analizar datos en \textit{streaming} y de forma \textit{batch}, mejorando la automatización, observabilidad y toma de decisiones en diversos casos de uso. 

Su arquitectura estándar en general incluye dispositivos y sensores para capturar datos, conectividad por red (Wi-Fi \cite{wifi}, 5G \cite{5g}, LoRaWAN \cite{lorawan}) para transmitirlos, una plataforma \textit{backend} formada por sistemas distribuidos en la nube para el almacenamiento, procesamiento y análisis de datos, y una interfaz de usuario para la visualización de resultados, pudiendo, en ocasiones, contar con sistemas de publicación y distribuición de eventos en \textit{streaming}. En general, las soluciones IoT generalmente estan organizadas en las siguientes capas:

\begin{itemize}

	\item Capa de percepción (\textit{Sensing Layer}): Es la capa más cercana al entorno físico y se encarga de la captura de datos. Los dispositivos IoT y sensores recolectan información del entorno, como temperatura, humedad, presión, movimiento, geolocalización, etc. Se implementan generalmente sistemas embebidos integrando sensores y actuadores conectados para poder interactuar con el ambiente físico.
	
	\item Capa de red (\textit{Network Layer}): Se encarga de la transmisión de datos desde los dispositivos hasta los sistemas de procesamiento. Aquí es donde ocurre la conectividad mediante diversos protocolos de comunicación usando tecnologías inalámbricas (tales como Wi-Fi, Bluetooth \citep{Bluetooth}, Zigbee \citep{zigbee}, LoRaWAN, NB-IoT \citep{Narrowband_IoT}, etc.) y protocolos de red (por ejmplo, MQTT, CoAP, HTTP, etc.)
	
	\item Capa de procesamiento o Borde (\textit{Edge Computing Layer}): Procesa datos cerca de donde se generan para reducir la latencia y el tráfico hacia la nube. Se toman decisiones inmediatas, y solo los datos relevantes se envían a niveles superiores. 
	
	\item Capa de almacenamiento y procesamiento \textit{cloud} (\textit{Data Storage/Cloud Layer}): Almacena y procesa grandes volúmenes de datos recolectados en la nube permitiendo realizar análisis más profundos, modelado de datos y aprendizaje automático. Se utilizan herramientas \textit{cloud} (como AWS IoT Core \citep{aws_iot_core}, Azure Iot Hub \citep{azure_iot}, etc.) y Big Data.
	
	\item Capa de aplicación (\textit{Application Layer}): Es la interfaz que permite a los usuarios interactuar con el sistema IoT. Aquí se presentan los datos de manera visual o se automatizan acciones basadas en la información recibida. Se utilizan tecnologías web y mobile, orientadas a eventos \textit{streaming} o dashboards para visualizar reportes \textit{batch}.
	
	\item Capa de seguridad (\textit{Security Layer}): Esta capa es transversal a los layers anteriores y tiene como función segurar la protección de datos, dispositivos y redes en todas las capas del sistema IoT. Es fundamental para evitar vulnerabilidades y ataques. Utiliza algoritmos de encriptación (como por ejemplo TLS/SSL y AES), protocolos de seguridad (como OAuth, OpenID Connect, etc).
	
\end{itemize}


Existen casos de uso de IoT en los cuales se utilizan robots exploradores como dispositivos físicos para la recopilación de datos en la capa de percepción. Los robots exploradores son dispositivos robotizados capaces de moverse de forma autónoma y/o controlados a distancia que utilizan sensores avanzados, inteligencia artificial y comunicación en tiempo real para navegar y monitorear condiciones ambientales en entornos peligrosos, como minas, plataformas petrolíferas, espacios confinados o áreas afectadas por desastres, entre otros. En agricultura, pueden inspeccionar cultivos; en medio ambiente, pueden monitorear la calidad del aire, del agua; en el espacio y océanos, son capaces de explorar lugares inaccesibles para el ser humano. 


Tanto en el ámbito académico como en la industria existen varios trabajos, proyectos, e implementaciones comerciales de soluciones IoT utilizando robots para mejorar la seguridad, la eficiencia y la toma de decisiones basada en datos, como por ejemplo: 

\begin{itemize}
	\item El desplegado en Lotus Mountain \cite{iot_usecase_seg_china} en Jilin, China para la seguridad en estaciones de esquí, implementado por perros robóticos equipados con sensores y tecnología de imágenes 3D que patrullan las pistas identificando peligros como desprendimientos y bloqueos en las rutas, mejorando así la seguridad de los esquiadores.

	\item El implementado por el Ayuntamiento de Bilbao \cite{iot_usecase_bilbao} para la inspección y mantenimiento de redes de saneamiento, que por medio de drones y robots, busca mejorar la eficiencia operativa y la seguridad de los trabajadores al reducir la necesidad de intervenciones humanas en entornos subterráneos y potencialmente peligrosos.

	\item El proyecto Tecnobosque \cite{iot_usecase_cuenca} en Cuenca, España, que utiliza drones equipados con sensores e inteligencia artificial para crear cortafuegos preventivos y reducir significativamente las hectáreas de bosques en casos de incendios. 


	\item Spot \cite{spot}, desarrollado por Boston Dynamics, un robot explorador cuadrupedo de propósito general capaz de explorar, almacenar y enviar información en tiempo real.
	  
	\item BIKE \cite{bike_inspection}, desarrollado por Waygate Technologies, un robot con ruedas magnéticas, muy utilizado en la industria de petróleo y gas entre otras, capaz de desplazarse por el interior de tuberías para poder realizar inspecciones y comunicar hallazgos.

	\item El prototipo robótico de exploración minera publicado en varios artículos \cite{latam-mining-robot-minero-unsj}, \cite{diario-de-cuyo-prototipo-robotico}, e impulsado por el Instituto de Automática de la Facultad de Ingeniería de la Universidad Nacional de San Juan en el marco de un convenio con la Comisión Nacional de Energía Atómica y el Gobierno argentino \cite{comunicacion-unsj-prototipo-convenio}.

	\item El robot de exploración terrestre denominado Geobot \cite{geobot} desarrollado por los ingenieros Nelson Dario García Hurtado y Melvin Andrés González Pino, de la universidad de Pamplona, capaz de realizar reconocimiento de zonas y manipulación de muestras de manera autónoma o asistida.

	\item El robot minero MIN-SIS 1.0 SDG-STR \cite{min-sis} desarrollado por los ingenieros Hernán L. Helguero Velásquez y Rubén Medinaceli Tórrez de la Universidad Técnica de Oruro, capaz de detectar gases, almacenar datos locales y enviar video e imágenes al puesto de mando.


\end{itemize}


En situaciones en las que es necesario explorar y monitorear un área ambientalmente sensible, como una reserva natural o un sitio afectado por un desastre ecológico y el objetivo es recopilar datos críticos, tales como niveles de contaminación, temperatura, humedad, y calidad del aire, es de gran importancia almacenar estas mediciones de una manera en la que se pueda asegurar la integridad y transparencia de los datos, como por ejemplo, en una cadena de bloques (blockchain).

Una arquitectura blockchain \cite{blockchain} se basa en el agrupamiento de transacciones que luego de ser procesadas, son almacenadas en bloques encadenados de forma distribuida e inmutable, entre los nodos de una red. Esta estructura de datos se conoce como una cadena de bloques y sus datos almacenados forma un \textit{distributed ledger} (o asiento contable distribuido). De esta manera, como los datos forman registros que no se pueden modificar una vez creados, se puede asegurar la inmutabilidad, y como el almacenamiento y procesamiento de la red se encuentran distribuidos, se puede garantizar su transparencia.

La mayoría de las redes blockchain constan de ciertas tecnologías para la implementación de código ejecutable en la misma red, que aunque su nombre puede cambiar dependiendo de la red, usualmente se los conoce como \textit{Smart Contracts} \citep{smart_contracts}. La ejecución de estos componentes es realizada por los nodos de la red en el proceso que se conoce como minería o validación. La forma de interactuar con los \textit{smart contracts} se realiza a través de otro componente conocido como dApps \textit{(de-centralized applications)} \citep{dapp} que haciendo uso de ciertas tecnologías invocan a estos componentes para almacenar y obtener datos en y desde el \textit{distributed ledger}.

La utilización de blockchain en arquitecturas IoT tiene varias ventajas, entre las que se pueden mencionar las vinculadas a las capacidades de descentralización, cifrado, seguridad y consenso que ayudan a aumentar la trazabilidad y transparencia de las operaciones, y la automatización de operaciones disparadas por eventos utilizando Smart Contracts.

Existen en la industria varias implementaciónes de casos de uso IoT en los que se ha utilizado blockchain como por ejemplo:

\begin{itemize}

	\item La solución basada en blockchain implementada por Walmart \citep{iot_usecase_blockchain_walmart} para mejorar la trazabilidad de productos alimenticios en su cadena de suministro. Al integrar dispositivos IoT, la empresa puede monitorear en tiempo real variables como temperatura y humedad durante el transporte y almacenamiento de productos perecederos. Estos datos se registran en una blockchain, garantizando la inmutabilidad y transparencia de la información.
	
\item La solución implementada por ScanTrust \citep{iot_usecase_blockchain_scantrust} que utiliza códigos QR seguros para conectar productos físicos con el entorno digital. Al integrar IoT y blockchain, permite a las empresas y consumidores autenticar productos y rastrear su origen y cadena de suministro en tiempo real. Los códigos QR, impresos en los envases, se escanean con dispositivos móviles, proporcionando información detallada y asegurando la autenticidad del producto.
		
	
	\item La solución implementada por la empresa Saltoki en colaboración con EcoMT \citep{iot_usecase_blockchain_saltoki}, que permite monitorizar y gestionar el consumo energético, certificando la producción renovable y los ahorros obtenidos mediante tecnología blockchain. 
	
	
\end{itemize}



%----------------------------------------------------------------------------------------

\section{Motivación del trabajo}


La motivación del presente trabajo fue primeramente volcar y unificar en un emprendimiento personal los conceptos aprendidos en la maestría de Internet de las Cosas. 

Se pretendió implementar una arquitectura robusta y flexible que pueda ser extrapolada a casos de uso en la industria en los que sea necesaria la integración de un sistema embebido con una red blockchain para el almacenamiento transparente e inmutable de datos sensibles.  

Por otra parte, se buscó desarrollar un producto cuya implementación pueda contribuir a aumentar el conocimiento público y el estado del arte de proyectos de código abierto para soluciones IoT integradas a blockchain en Argentina.

%----------------------------------------------------------------------------------------

\section{Alcance y objetivos}

A continuación, se detallan las funcionalidades incluidas en el alcance del trabajo.


\begin{itemize}
	\item La publicación del endpoint MQTT \citep{mqtt_spec} para la recepción de los datos enviados por el robot.
	\item La adaptación del sistema embebido del robot de exploración ambiental para la conexión segura con el \textit{backend} vía MQTT.
	\item La arquitectura e implementación de los sistemas \textit{backend} y el modelo de datos necesario para el almacenamiento de las mediciones enviadas por el robot.
	\item La arquitectura, implementación y despliegue de la dApp \citep{dapp} y \textit{smart contracts} \citep{smart_contracts} necesarios para el almacenamiento de las mediciones en una red Blockchain.
	\item La definición de métricas agregadas de valor y posterior arquitectura e implementación de los sistemas analíticos para procesar de forma \textit{batch} y/o \textit{real-time} utilizando herramientas de procesamiento paralelo basadas en Big Data.
	\item La implementación de la interfaz gráfica para poder visualizar los datos enviados y analíticas calculadas.

\end{itemize}



%----------------------------------------------------------------------------------------

\section{Requerimientos del producto}


A continuación, se listan los requerimientos del producto:

\begin{enumerate}	
	\item Requerimientos funcionales		
	\begin{enumerate}	
		
		\item El robot de exploración ambiental debe poder enviar a la plataforma datos de mediciones de parámetros ambientales, incluyendo los datos de fecha, hora, localización geográfica (que puede ser implementada como un valor \textit{mock}) y la categorización si es o no un valor crítico.
		\item El robot de exploración ambiental debe incorporar lógica para categorizar los valores medidos de cada parámetro ambiental como valores críticos si:
		\begin{enumerate}				
			\item Representan un máximo o mínimo global sensado hasta el momento.				
			\item Representan un máximo o mínimo local durante el último día.				
		\end{enumerate}			
		\item La solución a desarrollar debe poder recibir y almacenar las mediciones de parámetros ambientales enviadas por el robot.
		\item Los datos considerados críticos deben ser almacenados en un sistema inmutable.
		\item La solución a desarrollar debe poder procesar las mediciones de parámetros ambientales enviadas por el robot para generar métricas de valor para el usuario de negocio.		
		\item La solución a desarrollar debe brindar dos \textit{frontend} con interfaz web:
			\begin{enumerate}				
				\item El \textit{frontend} para el usuario de negocio.				
				\item El \textit{frontend} para el usuario administrador.				
			\end{enumerate}			
		
		\item El \textit{frontend} para el usuario de negocio debe proveer métricas para visualizar:
			\begin{enumerate}				
				\item Las lecturas históricas almacenadas.				
				\item Agregaciones (máximo, mínimo, promedio, etc.) de cada parámetro ambiental agrupado por frecuencias (ventanas de tiempo) y coordenadas geográficas.				
				\item Las referencias a los datos persistidos en blockchain.
			\end{enumerate}			
		\item El \textit{frontend} para el usuario de administración debe permitir:
			\begin{enumerate}				
				\item Acceder a los diferentes recursos utilizados por la herramienta (topics MQTT, \textit{smart contracts}, \textit{buckets}, etc.).
				\item Resetear valores y estado.			
			\end{enumerate}			
		\end{enumerate}	

					
	\item Requerimientos no funcionales		
	\begin{enumerate}	
		\item La solución a desarrollar debe contar con al menos un \textit{backend} de procesamiento y acceso a datos operacionales para la lógica de negocio.
		\item La solución a desarrollar debe contar con al menos un \textit{backend} de acceso, procesamiento, almacenamiento de datos analíticos para la generación de métricas.		
		\item El envío de los valores ambientales censados al \textit{backend} debe ser mediante MQTT.
		\item Las lecturas ambientales categorizadas como críticas deben ser almacenadas en blockchain para garantizar fiabilidad e inmutabilidad.
		\item La gestión de datos almacenados en blockchain debe ser implementada mediante \textit{smart contracts} desplegados en la red.
		\item La interacción con los \textit{smart contracts} debe realizarse desde una dApp.
		\item Los sistemas de transferencia y almacenamiento de datos utilizados deben contar con seguridad, permitiendo encriptación, autenticación y autorización.	
		\end{enumerate}	
		
	\item Requerimientos de documentación		
		\begin{enumerate}			
			\item Video demostrativo.	
			\item Documentación de arquitectura técnica del diseño del sistema.			
			\item Manual de usuario.	
			\item Memoria final.	
		\end{enumerate}	
		

		
		
	\item Requerimientos de testing		
		\begin{enumerate}			
			\item Se deben incluir tests de unitarios de componentes.
			\item Se deben incluir tests funcionales (\textit{smoke test}) del producto general.		
		\end{enumerate}	
	
	\item Requerimientos opcionales		
		\begin{enumerate}			
			\item De infraestructura y despliegue:
				\begin{enumerate}			
					\item Se permite realizar el despliegue de la dApp en un IPFS (preferentemente) o en la nube.					
					\item Se permite la incorporación de nuevo hardware al robot para la captura de datos adicionales.
					\item Se permite agregar automatización para la creación de la infraestructura como código.
				\end{enumerate}			
			
			\item De datos:
				\begin{enumerate}			
					\item Se permite almacenar cualquier otro dato adicional sensado o derivado.
					\item Se permite agregar cualquier implementación de gobierno de datos.	
					\item Se permite almacenar cualquier otra métrica o gráfico de explotación de datos adicional.
				\end{enumerate}
		
	\end{enumerate}
\end{enumerate}
\chapter{Introducción específica} % Main chapter title

\label{Chapter2}



%----------------------------------------------------------------------------------------
%	SECTION 1
%----------------------------------------------------------------------------------------
En este capítulo se presenta una breve introducción técnica a las herramientas de hardware y software utilizadas en el trabajo.

\section{Tecnologías de hardware y firmware utilizadas}


\subsection{Robot de exploración ambiental}

Como dispositivo físico en la capa de percepción, se utilizó el robot de exploración ambiental desarrollado en el marco de la carrera de Especialización de Sistemas Embebidos \citep{cese_gonzalo_memoria}. En la figura \ref{fig:Robot_y_Joystick_1} se puede apreciar una foto del mismo.


\begin{center}
   \includegraphics[scale=0.5]{Robot_y_Joystick_1}
   \captionof{figure}{Robot de exploración ambiental.}
   \label{fig:Robot_y_Joystick_1}
\end{center}

El robot de exploración ambiental es un sistema embebido desarrollado sobre \cite{ESP32} utilizando el marco de desarrollo ESP-IDF \cite{ESPIDF_home} de \textit{Espressif Systems} y FreeRTOS (Free Real-Time Operating System) \citep{FreeRTOS} como sistema operativo. Sus principales funciones son la exploración de terrenos de forma controlada por joystick y el la obtención de ciertos parámetros ambientales como luminosidad, presión atmosférica, temperatura y humedad.



\section{Tecnologías backend utilizadas}


\subsection{Amazon Web Services}

AWS (\textit{Amazon Web Services}) \citep{aws}  es una de las principales plataformas de servicios en la nube pública proporcionada por Amazon, que ofrece una amplia gama de productos y herramientas para computación, almacenamiento, bases de datos, redes, inteligencia artificial, seguridad y herramientas de desarrollo.


\subsection{AWS App Runner}

AWS App Runner \citep{aws_app_runner} es un servicio completamente administrado de AWS que permite implementar y ejecutar aplicaciones web y servicios de forma rápida, sin tener que gestionar instancias de infraestructura. Está diseñado para simplificar el proceso de implementación y escalado automático de aplicaciones y permite aplicaciones directamente desde el código fuente o desde contenedores Docker.



\subsection{AWS Glue}

AWS Glue \citep{aws_glue} es un servicio totalmente administrado de AWS diseñado para facilitar la extracción, transformación y carga de datos (ETL) en la nube que permite a los usuarios descubrir, preparar y combinar datos de múltiples fuentes para su análisis y almacenamiento en data lakes, data warehouses o bases de datos. Resulta útil para proyectos de Big Data y análisis de datos ya que brinda un servicio de catálogo de datos, asistencia para la generación de código ETL y un servicio de ejecución de trabajos de procesamiento paralelo.


\subsection{AWS S3}

AWS S3 (\textit{Simple Storage Service}) \citep{aws_s3} es un servicio de almacenamiento de objetos proporcionado por AWS totalmente administrado. Está diseñado para almacenar y recuperar cualquier cantidad de datos desde cualquier lugar, de forma segura, escalable y económica. Es ideal para almacenamiento de datos no estructurado de grandes volúmenes, sitios web estáticos, archivos multimedia, copias de seguridad, etc.

\subsection{AWS Athena}

AWS Athena \citep{aws_athena} es un servicio de análisis de datos serverless proporcionado por AWS que permite consultar directamente datos almacenados en Amazon S3 y los esquemas definidos en Glue, utilizando SQL estándar sin necesidad de configurar ni administrar servidores. Es ideal para analizar grandes volúmenes de datos de forma rápida y económica.



\subsection{MQTT}

MQTT (\textit{Message Queuing Telemetry Transport}) \citep{mqtt_spec} es un protocolo de comunicación asincrónico, ligero y orientado a mensajes, diseñado específicamente para dispositivos con recursos limitados y redes de baja ancho de banda. Es ampliamente utilizado en casos de uso IoT para la transmisión de datos en tiempo real entre dispositivos, sensores y aplicaciones. Utiliza un modelo de comunicacion basado en publish/subscribe y una estructura de datos basada en topics a los cuales los componentes clientes se conectan a un servicio broker para publicar o recibir notificaciones. 

\subsection{AWS IoT Core}

AWS IoT Core \citep{aws_iot_core} es un servicio totalmente administrado de AWS que permite conectar dispositivos IoT (Internet of Things) a la nube de manera segura y confiable. Proporciona una infraestructura escalable para recopilar, procesar y analizar datos de dispositivos en tiempo real, así como para interactuar con otros servicios de AWS a los que se redirigen los mensajes recibidos mediante la configuración de reglas. Tiene soporte para varios protocolos de comunicaciones, entre los que se destacan principalmente MQTT, HTTP/S, WebSockets y LoRaWAN.

\subsection{Node.js}

Node.js \citep{nodejs} es un entorno de ejecución de código abierto, construido sobre el motor de JavaScript V8 de Google Chrome. Aunque debido a su diseño puede ser utilizado para desarrollar aplicaciones backend de propósito general que requieran escalabilidad y rendimiento, es utilizado principalmente como servidor web. Para su funcionamiento utiliza un modelo single-thread de event loop con I/O no bloqueante, por lo que gestiona un bucle de eventos encolados y los procesa invocando sus callbacks de forma asincrónica sin realizar bloqueo de entradas y salidas en los puertos de comunicaciones, permitiendo atención de multiples solicitudes y paralelismo de tareas.

\section{Tecnologías Blockchain utilizadas}


\subsection{Ecosistema Ethereum}

Ethereum \citep{ethereum} es una red blockchain pública diseñada para el procesamiento de transacciones de forma descentralizada con almacenamiento distribuido, inmutable y de acceso libre ( \textit{permissionless}). La red se encuentra formada por los nodos de procesamiento, tambien denominados validadores, que tienen como función procesar transacciones y como otras redes blockchain, utiliza una estructura de datos basada en una cadena de bloques, en los cuales se van agrupando las transacciones validadas. 

Ethereum tiene como \textit{token} el Ether cuyo símbolo es ETH y tiene varios usos, pudiendo ser utilizado como criptomoneda de cambio y ahorro entre los usuarios finales de la red, pero también para pagar el gas (costo de ejecución de transacciones y smart contracts) y los fees a los validadores.

El proceso de validación de transacciones y generación de bloques, a partir de la versión 2.0 de Ethereum, utiliza el protocolo PoS (Proof-of-Stake) \citep{PoS} y opera en ranuras de tiempo llamadas slots, con un bloque propuesto aproximadamente cada 12 segundos. 

El primer paso de este proceso es la recolección de transacciones enviadas por los usuarios a la red (por ejemplo, para transferir dinero o ejecutar contratos) y su almacenamiento en un \textit{mempool} temporal. Luego, un validador es seleccionado de forma aleatoria para proponer el siguiente bloque, seleccionando del \textit{mempool} aquellas transacciones con tarifas de gas mas altas para maximizar su recompensa. El validador seleccionado incluye: las transacciones válidas, estado actualizado del sistema, hash del bloque anterior y los datos adicionales como la firma del bloque. Como resultado, este bloque es propuesto al resto de la red, en la que posteriormente, un comité de validadores, seleccionado de forma aleatoria, revisa el bloque verificando que las transacciones sean válidas, el bloque no esté duplicado o malicioso y sea coherente con el estado de la blockchain. Si el bloque es válido, los validadores emiten un voto (attestation) que confirma su aprobación. Si más de 2/3 de los validadores en el comité atestiguan el bloque, se considera finalizado y el bloque es agregado a la cadena de bloques de forma permanente.

El proceso de compensación y penalización de PoS retribuye a los validadores por diferentes acciones, con el fin de mantener la seguridadm, integridad y consenso de la red, aplica una técnica llamada slashing para la penalización por acciones malisiosas o incorrectas durante el procesamiento. El validador que propone el bloque recibe recompensas por bloque y tarifas de gas. Los validadores que votan correctamente para validar bloques también reciben recompensas proporcionales a su participación. Si el validador no presenta comportamiento malicioso o inactividad, no es penalizado. Sin embargo, los validadores pueden perder parte o todo su capital en stak si proponen múltiples bloques en un mismo slot, o votan de manera inconsistente (por ejemplo, intentando atacar la red), o están inactivos durante largos períodos de tiempo.

Antes de la versión 2.0 de Ethereum, se utilizaba otro protocolo de consenso llamado PoW (Proof-of-Work) \citep{PoW}, en el cual los nodos validadores desempeñaban el rol de mineros que competían por la generación del bloque, recompensando al que lo lograba generar y desaprovechando los recursos de cómputo utilizados por los que no lo lograron. El protocolo PoS en la version actual de Ethereum tiene varias ventajas con respecto a PoW consumiendo un 99,9 \% de energía, aumenta la escalabilidad con técnicas de sharding, y reduce las barreras de entrada al no requerir disponer de un hardware costoso para poder participar del proceso de validación.

% *intro a la evm*

Ethereum se diferencia de otras blockchains, como Bitcoin, porque no es solo un libro mayor digital, sino también una plataforma programable en la cual utilizando un SDK se pueden contruir programas denominados \textit{Smart Contracts} que se despliegan y ejecutan en la red. Como se mencionó anteriormente, los Smart Contracts (o contratos inteligentes) son programas informáticos autónomos que se ejecutan en redes blockchain como Ethereum, Solana o Binance Smart Chain. y están diseñados para automatizar, verificar y hacer cumplir acuerdos sin necesidad de intermediarios. Funcionan bajo el principio de si sucede una condición, entonces ejecutar una acción, aunque también pueden ser invocados de forma directa para operaciones de lectura y escritura. Una vez desplegados en la blockchain, no se pueden modificar, y todas las transacciones quedan registradas públicamente. 

El ciclo de vida de los \textit{smart contracts} comienza con su desarrollo utilizando alguno de los lenguajes de programación y SDK disponibles, como por ejemplo Solidity y Truffle. Una vez desarrollado, tras el proceso de compilación se obtiene un ABI \citep{abi} o especificación del contrato en formato JSON que no es enviado a la blockchain, sino que tiene como proposito poder ser utilizado posteriormente para acceder a los atributos y métodos del mismo a la hora de invocarlo. Posteriormente, durante el proceso de despliegue, se genera un binario del contrato que es enviado a la blockchain y tras ser procesado como una transacción mas, queda disponible en la red de manera inmutable. Al estar desplegado y disponible puede ser invocado o ejecutado automaticamente cuando se cumplen ciertas condiciones, generando nuevas transacciónes inmutables procesadas por la red y disponibles para ser consultadas.

Como se mencionó anteriormente, para la invocación a los \textit{smart contracts} se utilizan las dApps como componente de abstracción. Las dApps son aplicaciones que se ejecutan fuera de la blockchain (pudiendo ser por ejemplo un \textit{backend} en Node.js o un \textit{frontend} Javascript) e interactúan con la blockchain a través de ciertas bibliotecas, como por ejemplo, Web3.js \citep{web3}. Para poder acceder a la red, y posteriormente invocar el \textit{smart contract}, la dApp necesita utilizar un endpoint RPC publicado por cualquier nodo de la red y disponer de la especificación ABI del \textit{smart contract} obtenido durante su compilación (y no se disponible en la red).

Ethereum consta de varias redes disponibles para distintos propósitos. La Mainnet \citep{mainnet} es su red principal para usos productivos. Además existen múltiples \textit{testnets}, o redes de prueba, como Sepolia \citep{sepolia} y Holesky \citep{holesky} entre otras, disponibles para ser usadas durante el desarrollo y evaluación de soluciones blockchain en entornos productivos sin necesidad de pagar con fondos reales. Cada red tiene sus propias características y configuración de parámetros como el consenso, gas \textit{fees} y emisión de bloques.


\subsection{Solidity}

Solidity \cite{solidity} es un lenguaje de programación de alto nivel, orientado a contratos inteligentes, específicamente diseñado para funcionar en la Ethereum Virtual Machine (EVM). Fue creado en 2014 por Gavin Wood, Christian Reitwiessner y otros desarrolladores de Ethereum. Su sintaxis es similar a JavaScript, Python y C++, lo que facilita el aprendizaje para desarrolladores familiarizados con esos lenguajes. 


\subsection{Biblioteca Web3.js}

web3.js \citep{web3} es una biblioteca de JavaScript que permite interactuar con la blockchain de Ethereum y otros protocolos compatibles con Ethereum Virtual Machine (EVM). Proporciona una forma sencilla de conectarse a nodos de Ethereum, realizar transacciones y leer datos de contratos inteligentes, directamente desde aplicaciones web o Node.js.


\subsection{Ganache}

Ganache \cite{ganache_website} es una herramienta de desarrollo de Ethereum que permite crear una blockchain local para probar, desarrollar y depurar contratos inteligentes y dApps de forma rápida y segura. Es parte del conjunto de herramientas de Truffle Suite y es ampliamente utilizada por desarrolladores para simular una red Ethereum sin necesidad de usar una red pública como Mainnet o Testnets (Goerli, Sepolia).


\subsection{Truffle}

Truffle \cite{truffle_website} es un framework de desarrollo para Ethereum y otras blockchains compatibles con EVM (Ethereum Virtual Machine). Es parte de Truffle Suite y proporciona herramientas para compilar, desplegar y probar contratos inteligentes, además de facilitar la gestión de proyectos basados en Web3. Truffle automatiza gran parte del proceso de desarrollo de dApps, reduciendo errores y mejorando la eficiencia.

\subsection{Alchemy}

Como se mencionó mas arriba, la dApp, implementada como un servicio Node.js, es la responsable de invocar al Smart Contract desplegado en Ethereum utilizando un endpoint RPC publicado por cualquier nodo de la red. Debido a que los nodos públicos de la red pueden resultar limitados por motivos de seguridad, rendimiento y confiabilidad, resulta una mejor alternativa levantar un nodo EVM administrado o consumir esto como un servicio de un proveedor como Alchemy. Alchemy \cite{alchemy_website} es una plataforma de desarrollo blockchain que proporciona herramientas e infraestructura para crear y gestionar dApps en Ethereum y otras redes compatibles con EVM. Es conocida como el "AWS de Blockchain" debido a que ofrece nodos como servicio y herramientas para facilitar la interacción con la blockchain sin necesidad de que los desarrolladores configuren y mantengan sus propios nodos. En el trabajo actual, se utilizó Alchemy como punto de integración entre la dApp y los Smart Contracts.

\subsection{Etherscan}

Etherscan \cite{etherscan} es un explorador de bloques y plataforma de análisis para la red Ethereum. Permite a los usuarios buscar, verificar y rastrear transacciones, contratos inteligentes, direcciones de billeteras y otros datos en tiempo real. Es una herramienta fundamental para los desarrolladores y usuarios de Web3, ya que ofrece transparencia y acceso abierto a la información almacenada en la blockchain de Ethereum, tanto la Mainnet como las redes de prueba (Sepolia, Holesky, etc).

\subsection{Metamask}

MetaMask \cite{metamask} es una billetera digital y extensión de navegador (también disponible como aplicación móvil) que permite a los usuarios interactuar con blockchains basadas en Ethereum y otros ecosistemas compatibles con Ethereum, como Binance Smart Chain (BSC) y Polygon. Es una herramienta fundamental para interactuar con aplicaciones descentralizadas (dApps), contratos inteligentes y realizar transacciones de criptomonedas directamente desde tu navegador.
En el presente trabajo se utilizó para almacenar los fondos en ETH obtenidos a través de faucets necesarios para pagar el gas de las transacciones.

\section{Tecnologías de desarrollo utilizadas}


%\begin{center}
%\end{center}
%\includegraphics[scale=0.25]{espressif}

% \footnotetext{Imagen tomada de \cite{espressif-website-esp-idf}}

\subsection{Plataforma Docker}

Docker \cite{docker_website} es un proyecto de código abierto que automatiza el despliegue de aplicaciones dentro de contenedores de software, proporcionando una capa adicional de abstracción y automatización de virtualización de aplicaciones en múltiples sistemas operativos. Docker utiliza características de aislamiento de recursos del kernel Linux, tales como cgroups y espacios de nombres (namespaces) para permitir que contenedores livianos independientes se ejecuten en paralelo de manera aislada evitando la sobrecarga de iniciar y mantener máquinas virtuales.

%\includegraphics[scale=0.15]{docker}



\subsection{Plataforma de CI/CD}
Durante el proceso de desarrollo del producto se utilizó CI/CD (\textit{continuous integration / continuous delivery}) mediante la integración de las siguientes herramientas:

\begin{itemize}
	\item Github \cite{SoftwareTool_Github}: servicio de repositorio y control de versiones de código fuente.
	\item AWS CodePipeline \cite{SoftwareTool_codePipeline}: servicio de compilación, empaquetado y ejecución \textit{builds}.
	\item AWS Elastic Container Registry \cite{SoftwareTool_ECR}: servicio de repositorio y control de versiones de imágenes Docker.
\end{itemize}

El objetivo de esta configuración de servicios es permitir que por cada cambio en el código fuente versionado en el controlador de versiones Github, se dispare un proceso de compilación y ejecución de tests unitarios notificando en tiempo real si dicho cambio agrega o no una falla al actual estado del desarrollo. En caso de pasar satisfactoriamente la compilación y ejecución de los tests entonces se genera una nueva imagen Docker con la última versión del codigo compilado y se versiona en Artifact Registry.

\subsection{Visual Studio Code}

Visual Studio Code \cite{vscode_website} es un editor de código fuente desarrollado por Microsoft para Windows, Linux, macOS y Web. Incluye soporte para la depuración, control integrado de Git, resaltado de sintaxis, finalización inteligente de código, fragmentos y refactorización de código.

%\includegraphics[scale=0.15]{vscode}

\subsection{Sistema operativo Ubuntu}
Ubuntu \cite{ubuntu_website} es una distribución Linux basada en Debian GNU/Linux y patrocinado por Canonical, que incluye principalmente software libre y de código abierto. Puede utilizarse en ordenadores y servidores, está orientado al usuario promedio, con un fuerte enfoque en la facilidad de uso y en mejorar la experiencia del usuario.

%\includegraphics[scale=0.25]{ubuntu}




\chapter{Diseño e implementación} % Main chapter title

\label{Chapter3} % Change X to a consecutive number; for referencing this chapter elsewhere, use \ref{ChapterX}

En este capítulo se presentan los detalles técnicos de diseño e implementación de la solución IoT que se tuvieron en cuenta durante el desarrollo del trabajo.


\section{Arquitectura de software del sistema}


El sistema cuenta con una arquitectura robusta y flexible en la que se integra el dispositivo robótico de exploración ambiental \citep{cese_gonzalo_memoria} desarrollado en el marco de la Carrera de Especialización en Sistemas Embebidos, con un sistema \textit{back-end} desplegado en la nube pública \citep{nube_publica}, y una red Blockchain \cite{blockchain} a fin de poder asegurar la inmutabilidad y transparencia de las lecturas ambientales. 


\section{Hardware e infraestructura del sistema}
 
<A desarrollar>
 
\section{Integración de los módulos y subsistemas}




\subsection{Capa de percepción}


El desarrollo e integración de los componentes de esta capa consistió en la adaptación del firmware desplegado en el robot explorador para extender sus funcionalidades y enviar las lecturas de parámetros ambientales al \textit{topic} MQTT. Dentro de las funcionalidades que se le agregaron al robot explorador se encontraron:

\begin{itemize}
	\item Capturar fecha y hora local del sistema.
	\item Generación de coordenadas geograficas (con datos \textit{mock}).
	\item Conexión segura con tópico MQTT y envío de los datos generados.	
		
\end{itemize}

La configuración de la fecha y hora se realizó por medio del uso del servicio SNTP \citep{sntp} que permite la sincronización del hardware de una red con la fecha y hora provista por servicios externos en estandar en una zona horaria. Esta configuración se realizó incluyendo el encabezado \textbf{esp\_sntp.h} en el código del robot. Una vez realizado esto fue posible obtener la fecha y hora local invocando a la función \textit{localtime}.

La generación de las coordenadas geograficas con datos \textit{mock} se realizo mediante la generación de las ecuaciones \ref{eq:mock_lat} y \ref{eq:mock_long} a continuación.

\begin{equation}
	\label{eq:mock_lat}
	MockLat = \left( \frac{rand()}{RAND\_MAX} \right) \left( LAT\_MAX - LAT\_MIN \right) + LAT\_MIN
\end{equation}
                
\begin{equation}
	\label{eq:mock_long}
	MockLong = \left( \frac{rand()}{RAND\_MAX} \right) \left( LONG\_MAX - LONG\_MIN \right) + LONG\_MIN
\end{equation}

Finalmente, los datos capturados fueron enviados en formato JSON al tópico MQTT con la estructura de la tabla:


\begin{table}[h]
	\centering
	\caption[caption corto]{Tabla de objetos AWS}
	\begin{tabular}{l c c}    
		\toprule
		\textbf{Nombre del campo} & \textbf{Tipo del campo} & \textbf{Descripción}  \\
		\midrule
		deviceId & string & Id del dispositivo \\		
		type & string & Tipo de lectura \\		
		value & string & Valor de la lectura \\		
		geoLat & string & Latitud geográfica \\		
		geoLong & string & Longitud geográfica\\		
		date & string & Fecha \\		
		time & string & Hora \\		
		
		\bottomrule
		\hline
	\end{tabular}
	\label{tab:json_fields}
\end{table}


A continuación podemos apreciar un valor de ejemplo del objeto JSON enviado por el robot:

\begin{lstlisting}
{   	
	"deviceId": "12ad-dao23-ux23",
	"type": "Temperature",
	"value": "0.00",
	"geoLat": "-26.056772",
	"geoLong": "-64.014824",
	"date": "2025-04-1",
	"time": "11:23:59"
}
\end{lstlisting}

\subsection{Capa de red}

El desarrollo de los componentes de esta capa consistió en la publicación de un \textit{topic} MQTT desde el servicio AWS IoT Core y la configuración de la lógica de redirección y almacenamiento de los mensajes recibidos en AWS S3.


Para la conexión segura con el tópico MQTT se configuró el servicio AWS IoT Core, donde se creó una nueva instancia de un dispositivo remoto con el nombre ESP32. Una vez creado este dispositivo se descargaron e instalaron en el código del robot los certificados listados a continuación:




\begin{itemize}
	\item AmazonRootCA1.pem, renombrado a brokerCA.crt:	Es la autoridad certificadora que AWS usa para firmar certificados de sus servidores. El dispositivo lo necesita para verificar la identidad del servidor AWS IoT al conectarse.
	\item dev-certificate.pem.crt, renombrado a client.crt: Contiene la clave pública correspondiente a la clave privada (.key) y está firmado por AWS (o por una CA en la que AWS confía) para verificar la identidad del dispositivo.
	\item dev-private.pem.key, renombrado a client.key: Usada por el dispositivo para firmar su identidad durante la conexión TLS. Nunca se comparte ni se sube a AWS. Tu dispositivo la usa para autenticar su certificado (.crt).
		
\end{itemize}

Una vez realizada la configuración del servicio AWS IoT core e integrado el robot con el tópico, se probó la recepción de lecturas con el cliente de prueba provisto por AWS suscrito al tópico \textit{readings} como puede apreciarse en la figura \ref{fig:aws_iot_core_mqtt_test_2}.

\begin{center}
   \includegraphics[scale=0.35]{AWS/aws_iot_core_mqtt_test_2}
   \captionof{figure}{Prueba de recepción de mensajes MQTT.}
   \label{fig:aws_iot_core_mqtt_test_2}
\end{center}

Para el almacenamiento en AWS S3 de los mensajes recibidos, se configuro una \textit{routing rule} o regla de redirección en AWS IoT Core, indicando mediante una consulta con sintaxis SQL, que todos los mensajes recibidos en el tópico \textit{readings} deben almacernarce en el bucket S3 ceiot-exploratory-robot. En la figura \ref{fig:aws_iot_core_message_routing} puede apreciarse esta configuración.
 

\begin{center}
   \includegraphics[scale=0.4]{AWS/aws_iot_core_message_routing}
   \captionof{figure}{Configuración de redirección de mensajes MQTT.}
   \label{fig:aws_iot_core_message_routing}
\end{center}

Como resultado de la configuración realizada, los mensajes recibidos en MQTT fueron redirigidos y almacenados en AWS S3 como puede apreciarse en la figura \ref{fig:aws_s3_bucket_data2}.

\begin{center}
   \includegraphics[scale=0.4]{AWS/aws_s3_bucket_data2}
   \captionof{figure}{Almacenamiento de mensajes JSON en AWS S3.}
   \label{fig:aws_s3_bucket_data2}
\end{center}


\subsection{Capas de procesamiento y almacenamiento - cloud}

Una vez realizadas las configuraciones de ingesta de datos en \textit{streaming} se realizaron las configuraciones para poder administrarlos y procesarlos.
Para ello se crearon una base de datos y una tabla en AWS Glue para representar el esquema de datos almacenados en AWS S3 en formato JSON, como se puede apreciar en la figura \ref{fig:aws_glue_table_review}.

\begin{center}
   \includegraphics[scale=0.4]{AWS/aws_glue_table_review}
   \captionof{figure}{Creación de base datos, tabla y esquema AWS Glue.}
   \label{fig:aws_glue_table_review}
\end{center}

Con el esquema de datos definido en el catálogo de AWS Glue, fue posible realizar consultas SQL sobre los datos almacenados en AWS S3 desde AWS Athena, como se puede apreciar en la figura 

\begin{center}
   \includegraphics[scale=0.4]{AWS/aws_athena_config}
   \captionof{figure}{Consulta de datos SQL desde AWS Athena.}
   \label{fig:aws_athena_config}
\end{center}





\subsection{Capas de procesamiento y almacenamiento - blockchain}

Uno de los primeros pasos para poder comenzar a desarrollar los componentes blockchain fue la obtención de tokens para poder realizar despliegues y ejecutar la aplicación en las redes de prueba de Ethereum sin utilizar fondos reales. Para Para poder realizar esto, primero fue necesario crear un \textit{wallet} o billetera digital, para lo que se utilizó el servicio Metamask. Luego, para la obtención de créditos se utilizaron los \textit{faucets} de Google \citep{google_faucets}. Como se puede apreciar en las figura \ref{fig:google_faucets2} y tras seleccionar la dirección del \textit{wallet} y \ref{fig:metamask_balance}, tras realizar las transacciones en el \textit{faucet} se reciben los fondos en la billeta digital. 

\begin{center}
   \includegraphics[scale=0.35]{blockchain/google_faucets2}
   \captionof{figure}{Obtención de créditos mediante Google Web3.}
   \label{fig:google_faucets2}
\end{center}

\begin{center}
   \includegraphics[scale=0.35]{blockchain/metamask_balance}
   \captionof{figure}{Saldo en Metamask.}
   \label{fig:metamask_balance}
\end{center}

Posteriormente como podemos apreciar en la figura, en el servicio Etherscan las transacciones realizadas para generar fondos en la dirección de la billetera quedan publicadas en la red.


\begin{center}
   \includegraphics[scale=0.3]{blockchain/sepolia_funding2}
   \captionof{figure}{Transacciones de generación de fondos de prueba.}
   \label{fig:sepolia_funding2}
\end{center}


Una vez obtenidos los fondos de prueba en la billetera se procedió con el desarrollo de los componentes blockchain. Para el desarrollo de los \textit{smart contracts} se utilizó Solidity como lenguaje de programanción y Truffle como herramienta de gestión de configuración, compilación, empaquetado y despliegue. Truffle utiliza una configuración basada en archivos Javascript para la descripción de las tareas, y para realizar el despliegue a diferentes redes, como por ejemplo de forma local a Ganache o de forma remota a redes como Sepolia, Holesky y Mainnet. En la los archivos de configuración de Truffle se agregaron entradas para poder desplegar a Ganache y a Sepolia como se puede apreciar en la figura \ref{fig:truffle_redes}.

\begin{center}
   \includegraphics[scale=0.4]{blockchain/truffle_redes}
   \captionof{figure}{Configuración de redes de despliegue en Truffle.}
   \label{fig:truffle_redes}
\end{center}


Desde el punto de vista del \textit{backend} blockchain, se desarrolló la dApp utilizando Node.js y la biblioteca Javascript Web3.js para la comunicación con los \textit{smart contracts}. Para la dApp, se desarrollaron varios \textit{endpoints} listados a continuacion en la siguiente tabla:



Como se puede apreciar en la figura \ref{fig:sm_deployment}, durante el proceso de despliegue se pueden observar cierta información por pantalla:

\begin{itemize}
	\item La red a la cual se esta realizando el despliegue.
	\item Los \textit{smart contracts} incluidos en el despliegue y la dirección que toman una vez desplegados. 
	\item El \textit{hash} o identificador de la transacción.
	\item El número de bloque en el que se encuentra la transacción de despliegue.
	\item La dirección de la billetera digital y el balance disponible previo a la transacción.
	\item La cantidad de gas utilizado en el despliegue, el costo unitario y el costo total de la transacción.
\end{itemize}

\begin{center}
   \includegraphics[scale=0.35]{blockchain/sm_deployment}
   \captionof{figure}{Salida por pantalla durante el proceso de despliegue de los componentes blockchain.}
   \label{fig:sm_deployment}
\end{center}

Luego de haber realizado el despliegue de los \textit{smart contracts} a la red Sepolia, se puede evidenciar desde Etherscan las transacciones reportadas en la salida por pantalla (figura \ref{fig:sm_deployment_etherscan1}) y comparar los valores. Como se puede apreciar en la figura \ref{fig:sm_deployment_etherscan2}, al ingresar a los detalles de la transacción podemos obtener mas información del contrato.

\begin{center}
   \includegraphics[scale=0.35]{blockchain/sm_deployment_etherscan1}
   \captionof{figure}{Transacciones de despliegue en Sepolia.}
   \label{fig:sm_deployment_etherscan1}
\end{center}

\begin{center}
   \includegraphics[scale=0.35]{blockchain/sm_deployment_etherscan2}
   \captionof{figure}{Detalles del contrato desplegado en Sepolia.}
   \label{fig:sm_deployment_etherscan2}
\end{center}


\section{Plataforma de desarrollo y despliegue}



\section{Tabla de todos los objetos AWS creados}




\begin{table}[h]
	\centering
	\caption[caption corto]{Tabla de objetos AWS}
	\begin{tabular}{l c c}    
		\toprule
		\textbf{Servicio} & \textbf{Propósito} & \textbf{Nombre de objeto}  \\
		\midrule
		AWS IoT Core & \textit{Thing} & ESP32 \\		
		AWS IoT Core & \textit{MQTT topic} & readings \\		
		AWS IoT Core & \textit{Routing Rule} & StoreToS3 \\		
		AWS S3 & Bucket & ceiot-exploratory-robot \\	
		AWS Glue & \textit Base de datos & ceit \\		
		AWS Glue & Tabla & \textit{Readings} \\		

		\bottomrule
		\hline
	\end{tabular}
	\label{tab:peces}
\end{table}


% Chapter Template

\chapter{Ensayos y resultados} % Main chapter title

\label{Chapter4} 

En este capítulo se describe el proceso de verificaciones y validaciones que se realizó a fin de comprobar el correcto funcionamiento del sistema y el alcance de los objetivos del trabajo.

%----------------------------------------------------------------------------------------
%	SECTION 1
%----------------------------------------------------------------------------------------

\section{Verificaciones técnicas}



\subsection{Verificación del set-up de dependencias Ethereum}

Como se mencionó anteriormente para poder interactuar con la red Ethereum se necesita disponer de saldo en ETH en un wallet y para esto se utilizaron Faucets de Google. En la siguiente figura \ref{fig:google_faucets2} se puede apreciar la verificación de la obtención de tokens.

\begin{center}
   \includegraphics[scale=0.35]{blockchain/google_faucets2}
   \captionof{figure}{Obtención de créditos mediante Google Web3.}
   \label{fig:google_faucets2}
\end{center}

En la figura \ref{fig:metamask_balance} se puede verificar la acreditación de los fondos obtenidos en la billetera Sepolia utilizada para las pruebas. 

\begin{center}
   \includegraphics[scale=0.35]{blockchain/metamask_balance}
   \captionof{figure}{Saldo en Metamask.}
   \label{fig:metamask_balance}
\end{center}

Posteriormente se pudo verificar desde Etherscan las transacciones realizadas en la transferencia de los fondos via faucets \ref{fig:sepolia_funding2}.

\begin{center}
   \includegraphics[scale=0.3]{blockchain/sepolia_funding2}
   \captionof{figure}{Transacciones de generación de fondos de prueba.}
   \label{fig:sepolia_funding2}
\end{center}

Como se mencionó anteriormente, el acceso a la red Ethereum se realizó a través del servicio Alchemy. En la siguiente figura \ref{fig:alchemy1} se puede apreciar la verificación de su set-up para poder ser invocado por la dApp. 


\begin{center}
   \includegraphics[scale=0.3]{blockchain/alchemy1}
   \captionof{figure}{Frontend the Alchemy.}
   \label{fig:alchemy1}
\end{center}




\subsection{Verificación del despliegue de los Smart Contracts}

Tras ejecutar el proceso de despliegue se pudo verificar la salida por consola con la confirmación del resultado exitoso. Como se puede apreciar en la figura \ref{fig:sm_deployment} se observa la dirección de los Smart Contracts, la cuenta, balance y gas utilizado.


\begin{center}
   \includegraphics[scale=0.35]{blockchain/sm_deployment}
   \captionof{figure}{Salida por pantalla durante el proceso de despliegue de los componentes blockchain.}
   \label{fig:sm_deployment}
\end{center}

Posteriormente, desde Etherscan se pueden apreciar las transacciones de cada despliegue en la red, por ejemplo en este caso Sepolia, con los detalles de cada una de las operaciones.

\begin{center}
   \includegraphics[scale=0.35]{blockchain/sm_deployment_etherscan1}
   \captionof{figure}{Transacciones de despliegue en Sepolia.}
   \label{fig:sm_deployment_etherscan1}
\end{center}

\begin{center}
   \includegraphics[scale=0.35]{blockchain/sm_deployment_etherscan2}
   \captionof{figure}{Detalles del contrato desplegado en Sepolia.}
   \label{fig:sm_deployment_etherscan2}
\end{center}



\subsection{Verificación del despliegue de la dApp}

Al realizar el despliegue por la consola web de AWS se puede apreciar el resultado exitoso del proceso. Como se observa en la figura \ref{fig:deployment_dapp}, tras el despliegue se puede obtener la URL desde la que se puede acceder a la dapp.

\begin{center}
   \includegraphics[scale=0.4]{AWS/deployment_dapp}
   \captionof{figure}{Despliegue de la dApp en App Runner.}
   \label{fig:deployment_dapp}
\end{center}

Como se puede apreciar en la siguiente figura \ref{fig:deployment_dapp_log} los logs del despliegue indican que el proceso se realizó exitosamente y se puede apreciar el historial de últimos despliegues.

\begin{center}
   \includegraphics[scale=0.4]{AWS/deployment_dapp_log}
   \captionof{figure}{Despliegue de la dApp en App Runner.}
   \label{fig:deployment_dapp_log}
\end{center}

Una vez verificado el despliegue en la consola de AWS se puede apreciar en la figura \ref{fig:dapp_endpoints} la verificación del funcionamiento de la dApp accediendo a su API Restful publicada por Swagger.


\begin{center}
   \includegraphics[scale=0.4]{blockchain/dapp_endpoints}
   \captionof{figure}{Endpoints expuestos por la dApp.}
   \label{fig:dapp_endpoints}
\end{center}


\subsection{Verificación de ingesta de datos en tiempo real MQTT}

Desde la herramienta AWS IoT Core se pudo verificar la recepción de mensajes MQTT desde el dispositivo ESP32 como se puede apreciar en la siguiente figura \ref{fig:aws_iot_core_mqtt_test_2}.

\begin{center}
   \includegraphics[scale=0.35]{AWS/aws_iot_core_mqtt_test_2}
   \captionof{figure}{Prueba de recepción de mensajes MQTT.}
   \label{fig:aws_iot_core_mqtt_test_2}
\end{center}


\subsection{Verificación del procesamiento de mensajes }

Una vez redireccionados desde AWS IoT Core por medio de AWS SNS, se pudo verificar el encolamiento de los mensajes en AWS SQS donde como se puede apreciar en la figura \ref{fig:events_sqs_2}, los mensajes se acumulan para ser procesados en la cola \textit{readings}.

\begin{center}
   \includegraphics[scale=0.4]{AWS/events_sqs_2}
   \captionof{figure}{Configuración de redirección de mensajes MQTT.}
   \label{fig:events_sqs_2}
\end{center}

Como se aprecia en la siguiente figura \ref{fig:events_sqs_1}, también se pudo verificar la tasa de invocaciones de AWS SQS desde su herramienta de monitoreo.

\begin{center}
   \includegraphics[scale=0.4]{AWS/events_sqs_1}
   \captionof{figure}{Configuración de redirección de mensajes MQTT.}
   \label{fig:events_sqs_1}
\end{center}


Como se mencionó anteriormente, desde la cola readings se dispara la ejecución de la función AWS Lambda que procesa los mensajes, invocando la dApp. Como se puede apreciar en la siguientes figuras \ref{fig:events_lambda1} y \ref{fig:events_lambda3}, se verificó el correcto funcionamiento del procesamiento de eventos.


\begin{center}
   \includegraphics[scale=0.4]{AWS/events_lambda1}
   \captionof{figure}{Monitoreo de la función AWS Lambda de procesamiento de eventos.}
   \label{fig:events_lambda1}
\end{center}

\begin{center}
   \includegraphics[scale=0.4]{AWS/events_lambda3}
   \captionof{figure}{Monitoreo de la función AWS Lambda de procesamiento de eventos.}
   \label{fig:events_lambda3}
\end{center}

\begin{center}
   \includegraphics[scale=0.4]{AWS/aws_s3_bucket_data2}
   \captionof{figure}{Almacenamiento de mensajes JSON en AWS S3.}
   \label{fig:aws_s3_bucket_data2}
\end{center}



\subsection{Verificación de la invocación de la dApp y los Smart Contracts}

En la dApp, se pudo verificar el correcto funcionamiento del sistema en sus herramientas de monitoreo como se puede apreciar en las siguientes figuras \ref{fig:events_dapp} y \ref{fig:events_cloudwatch_dapp} las métricas de request HTTP y los logs de las transacciones Ethereum.


\begin{center}
   \includegraphics[scale=0.4]{AWS/events_dapp}
   \captionof{figure}{Almacenamiento de mensajes JSON en AWS S3.}
   \label{fig:events_dapp}
\end{center}

\begin{center}
   \includegraphics[scale=0.4]{AWS/events_cloudwatch_dapp}
   \captionof{figure}{Logs en Cloudwatch de las transacciones.}
   \label{fig:events_cloudwatch_dapp}
\end{center}


\subsection{Validación de los datos almacenados en Ethereum}


Por cada ejecución tambien se pudo verificar en Etherscan la creación de nuevas transacciones tras la ejecución de la dApp, como se puede apreciar en la figura \ref{fig:events_etherscan}.

\begin{center}
   \includegraphics[scale=0.4]{AWS/events_etherscan}
   \captionof{figure}{Creación de nuevas transacciones en Etherscan.}
   \label{fig:events_etherscan}
\end{center}


\subsection{Verificación del proceso de almacenamiento de lecturas, transacciones y contratos en AWS S3}

Como se puede apreciar en las figuras se pudo verificar en AWS S3 el correcto almacenamiento de los objetos datos de lecturas, transacciones y contratos en el bucket configurado.

\begin{center}
   \includegraphics[scale=0.4]{AWS/events_s3_tx}
   \captionof{figure}{Almacenamiento de mensajes JSON en AWS S3.}
   \label{fig:events_s3_tx}
\end{center}


\begin{center}
   \includegraphics[scale=0.4]{AWS/events_s3_readings}
   \captionof{figure}{Almacenamiento de mensajes JSON en AWS S3.}
   \label{fig:events_s3_readings}
\end{center}


\begin{center}
   \includegraphics[scale=0.4]{AWS/events_s3_contracts}
   \captionof{figure}{Almacenamiento de mensajes JSON en AWS S3.}
   \label{fig:events_s3_contracts}
\end{center}


\subsection{Verificación del acceso a datos desde AWS Athena}



\begin{center}
   \includegraphics[scale=0.4]{AWS/aws_athena_config}
   \captionof{figure}{Consulta de datos SQL desde AWS Athena.}
   \label{fig:aws_athena_config}
\end{center}





\subsection{Verificación del proceso de ingesta y transformación de datos batch en Fabric}


\begin{center}
   \includegraphics[scale=0.35]{Azure/datafactory_result}
   \captionof{figure}{Correcta ejecución del pipeline punta-a-punta con Azure Data Factory y Dataflows.}
   \label{fig:powerbi1}
\end{center}
 

\subsection{Validación del modelo de datos en el Semantic Model}

\subsection{Validación del reporte final generado en PowerBI}

\begin{center}
   \includegraphics[scale=0.4]{Azure/powerbi}
   \captionof{figure}{Dashboard PowerBI.}
   \label{fig:powerbi1}
\end{center}
 

\section{Pruebas funcionales del sistema}

\include{Chapters/Chapter5}

%----------------------------------------------------------------------------------------
% Apéndices
%----------------------------------------------------------------------------------------

\appendix

% Incluir apéndices desde archivos separados si es necesario
%\include{Appendices/AppendixA}

%----------------------------------------------------------------------------------------
% Bibliografía
%----------------------------------------------------------------------------------------

\renewcommand{\bibname}{Bibliografía} % Para asegurarte de que el título sea correcto
\phantomsection % Necesario para que el enlace del marcador sea correcto

\printbibliography[heading=bibintoc]

\end{document}






