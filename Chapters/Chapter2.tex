\chapter{Introducción específica} % Main chapter title

\label{Chapter2}



%----------------------------------------------------------------------------------------
%	SECTION 1
%----------------------------------------------------------------------------------------
Esta sección presenta una breve introducción técnica a las herramientas hardware y software utilizadas en el trabajo.

\section{Tecnologías de hardware y firmware utilizadas}


\subsection{Robot de exploración ambiental Espressif ESP32}


ESP32 \cite{ESP32} es una serie de microcontroladores embebidos en un chip con Wi-Fi y Bluetooth integrados, de bajo costo y consumo, desarrollado por \textit{Espressif Systems}. Emplea dos cores Xtensa® 32-bit LX6 CPU, incluye interruptores de antena, amplificador de potencia, amplificador de recepción de bajo ruido, un co-procesador ULP (\textit{Ultra Low Power}), módulos de administración de energía y varios periféricos.
En la siguiente imagen (\ref{fig:esp32}) se puede apreciar la placa ESP32-WROOM-32D \cite{ESP32_wroom_32d_datasheet} utilizada para el desarrollo del presente trabajo.



\subsection{Marco de trabajo ESP-IDF}

\textit{Espressif Systems} proporciona recursos básicos de hardware y software para ayudar a los desarrolladores de aplicaciones a realizar sus ideas utilizando el hardware de la serie ESP32. El framework de software de Espressif está destinado al desarrollo de aplicaciones de IoT (Internet de las cosas) con Wi-Fi, Bluetooth, administración de energía y varias otras características del sistema.
Sus componentes son:
\begin{enumerate}
	\item Toolchain, utilizado para compilar el código para ESP32.
	\item Build tools, que provee utilidades como CMake \cite{cmake_website} y Ninja \cite{ninja_website} para construir la aplicación completa para ESP32.
	\item ESP-IDF \cite{ESPIDF_home}, que brinda la API de desarrollo para ESP32 y scripts para ejecutar Toolchain.
	
\end{enumerate}

Además de las herramientas mencionadas se utilizó el conjunto de bibliotecas y drivers provistos por el proyecto ESP-IDF-Lib \cite{esp_idf_lib_website} basados en el framework ESP-IDF.



\subsection{Sistema operativo FreeRTOS}
 FreeRTOS \citep{FreeRTOS} 



\subsection{Testing unitario}
Con el fin de maximizar la calidad durante el proceso de desarrollo del producto se implementaron test unitarios para todos los servicios del robot y del joystick. El conjunto de herramientas utilizadas para tal fin fueron:
\begin{itemize}
	\item Ceedling \cite{SoftwareTool_Ceedling}: herramienta de orquestacion de tests unitarios, inyección de objetos mocks.
	\item CMock \cite{SoftwareTool_CMock}: framework de mock objects para sistemas embebidos.
	\item Unity \cite{SoftwareTool_Unity}: framework de unit testing para sistemas embebidos.
	\item Gcov \cite{SoftwareTool_CeedlingGCov}: plugin the ceedling para evaluar y reportar la cobertura.
\end{itemize}

En los capítulos siguientes se describe la configuración de dichas herramientas y se presentan los resultados tras su ejecución.


\section{Tecnologías operacionales y analíticas utilizadas}


\subsection{Amazon Web Services}

blablabla



\subsection{AWS App Runner}

blablabla



\subsection{AWS Glue}

blablabla

\subsection{AWS Athena}

blablabla



\subsection{AWS S3}

blablabla


\subsection{MQTT}

blablabla


\subsection{Node}

blablabla

\section{Tecnologías Blockchain utilizadas}


\subsection{Ethereum}

blablabla

\subsection{Solidity}

blablabla


\subsection{Web3}

blablabla


\subsection{Ganache}

blablabla


\subsection{Truffle}

blablabla

\subsection{Alchemy}

blablabla


\subsection{Etherscan}

blablabla

\subsection{Metamask}

blablabla


\section{Tecnologías de desarrollo utilizadas}


%\begin{center}
%\end{center}
%\includegraphics[scale=0.25]{espressif}

% \footnotetext{Imagen tomada de \cite{espressif-website-esp-idf}}

\subsection{Plataforma Docker}

Docker \cite{docker_website} es un proyecto de código abierto que automatiza el despliegue de aplicaciones dentro de contenedores de software, proporcionando una capa adicional de abstracción y automatización de virtualización de aplicaciones en múltiples sistemas operativos. Docker utiliza características de aislamiento de recursos del kernel Linux, tales como cgroups y espacios de nombres (namespaces) para permitir que contenedores livianos independientes se ejecuten en paralelo de manera aislada evitando la sobrecarga de iniciar y mantener máquinas virtuales.

%\includegraphics[scale=0.15]{docker}



\subsection{Plataforma de CI/CD}
Durante el proceso de desarrollo del producto se utilizó CI/CD (\textit{continuous integration / continuous delivery}) mediante la integración de las siguientes herramientas:

\begin{itemize}
	\item Github \cite{SoftwareTool_Github}: servicio de repositorio y control de versiones de código fuente.
	\item Google Cloud Build \cite{SoftwareTool_CloudBuild}: servicio de compilación, empaquetado y ejecución \textit{builds}.
	\item Google Artifact Registry \cite{SoftwareTool_ArtifactRegistry}: servicio de repositorio y control de versiones de imágenes Docker.
\end{itemize}

El objetivo de esta configuración de servicios es permitir que por cada cambio en el código fuente versionado en el controlador de versiones Github, se dispare un proceso de compilación y ejecución de tests unitarios notificando en tiempo real si dicho cambio agrega o no una falla al actual estado del desarrollo. En caso de pasar satisfactoriamente la compilación y ejecución de los tests entonces se genera una nueva imagen Docker con la última versión del codigo compilado y se versiona en Artifact Registry.

\subsection{Visual Studio Code}

Visual Studio Code \cite{vscode_website} es un editor de código fuente desarrollado por Microsoft para Windows, Linux, macOS y Web. Incluye soporte para la depuración, control integrado de Git, resaltado de sintaxis, finalización inteligente de código, fragmentos y refactorización de código.

%\includegraphics[scale=0.15]{vscode}

\subsection{Sistema operativo Ubuntu}
Ubuntu \cite{ubuntu_website} es una distribución Linux basada en Debian GNU/Linux y patrocinado por Canonical, que incluye principalmente software libre y de código abierto. Puede utilizarse en ordenadores y servidores, está orientado al usuario promedio, con un fuerte enfoque en la facilidad de uso y en mejorar la experiencia del usuario.

%\includegraphics[scale=0.25]{ubuntu}



