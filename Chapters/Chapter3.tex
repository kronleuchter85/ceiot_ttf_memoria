\chapter{Diseño e implementación} % Main chapter title

\label{Chapter3} % Change X to a consecutive number; for referencing this chapter elsewhere, use \ref{ChapterX}

En este capítulo se presentan los detalles técnicos de diseño e implementación de la solución IoT que se tuvieron en cuenta durante el desarrollo del trabajo.


\section{Arquitectura de software del sistema}


El sistema cuenta con una arquitectura robusta y flexible en la que se integra el dispositivo robótico de exploración ambiental \citep{cese_gonzalo_memoria} desarrollado en el marco de la Carrera de Especialización en Sistemas Embebidos, con un sistema \textit{back-end} desplegado en la nube pública \citep{nube_publica}, y una red Blockchain \cite{blockchain} a fin de poder asegurar la inmutabilidad y transparencia de las lecturas ambientales. 


\section{Hardware e infraestructura del sistema}
 
 
\section{Integración de los módulos y subsistemas}

\subsection{Capas de percepción y red}

El desarrollo e integración de los componentes de estas capa consistió en la publicación de un \textit{topic} MQTT desde el \textit{backend} y la adaptación del firmware desplegado en el robot explorador para extender sus funcionalidades y enviar las lecturas ambientales al \textit{topic}. Para esto se configuró el servicio AWS IoT Core, donde se creó una nueva instancia de un dispositivo remoto con el nombre ceit-robot-explorador y se configuró el nombre del \textit{topic} como /ceit/robot/readings.

Se configuró la redirección de los mensajes MQTT recibidos a un \textit{bucket} AWS S3 para poder ser almacenados y se descargó el empaquetado de certificados de seguridad que deben ser desplegados en el robot explorador para poder conectarse al \textit{topic}.

\subsection{Capas de procesamiento y almacenamiento cloud}



\section{Plataforma de desarrollo y despliegue}




