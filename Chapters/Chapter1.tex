% Chapter 1

\chapter{Introducción general} % Main chapter title

\label{Chapter1} % For referencing the chapter elsewhere, use \ref{Chapter1} 
\label{IntroGeneral}

Esta sección presenta la motivación, alcance, objetivos y requerimientos del producto en el marco del estado del arte y su importancia en la industria.

%----------------------------------------------------------------------------------------

% Define some commands to keep the formatting separated from the content 
\newcommand{\keyword}[1]{\textbf{#1}}
\newcommand{\tabhead}[1]{\textbf{#1}}
\newcommand{\code}[1]{\texttt{#1}}
\newcommand{\file}[1]{\texttt{\bfseries#1}}
\newcommand{\option}[1]{\texttt{\itshape#1}}
\newcommand{\grados}{$^{\circ}$}

%----------------------------------------------------------------------------------------

%\section{Introducción}

%----------------------------------------------------------------------------------------
\section{Estado del arte}


Los robots exploradores son dispositivos robotizados capaces de moverse de forma autónoma, y/o controlados a distancia, que han sido creados con el fin de reconocer y explorar un lugar o entorno donde una persona no pueda o deba acceder ya sea por motivos de capacidad, practicidad o seguridad. Por este motivo, en función de las necesidades de desplazamiento, existen diferentes sistemas de motricidad, como son por ejemplo, los bípedos, cuadrúpedos, con ruedas, tracción oruga, acuáticos/sumergibles, aéreos, etc. En cuanto a la forma de control, los hay manejados por control remoto cableado o inalámbrico, habiendo equipos más sofisticados, que gracias a aplicaciones de Inteligencia Artificial, están preparados para desplazarse y tomar decisiones de forma autónoma. Algunos de los tipos de robots exploradores más conocidos son los espaciales, de minas, de rescate en catástrofes, de tuberías, submarinos, y de suelos.

Los robots exploradores suelen desempeñar el papel de dispositivo físico o sistema embebido, en casos de uso mas grandes como por ejemplo en soluciones IoT. En estos, existe uno o varios sistemas \textit{back-end} distribuidos a los cuales se integra el dispositivo robótico para comunicar, enviar y/o recibir datos.

Tanto en el ámbito académico como en la industria existen varios trabajos, proyectos, e implementaciones comerciales de robots exploradores, como por ejemplo: 

\begin{itemize}
	\item El desplegado en Lotus Mountain \cite{iot_usecase_seg_china} en Jilin, China para la seguridad en estaciones de esquí, implementado por perros robóticos equipados con sensores y tecnología de imágenes 3D que patrullan las pistas identificando peligros como desprendimientos y bloqueos en las rutas, mejorando así la seguridad de los esquiadores.

	\item El implementado por el Ayuntamiento de Bilbao \cite{iot_usecase_bilbao} para la inspección y mantenimiento de redes de saneamiento, que por medio de drones y robots, busca mejorar la eficiencia operativa y la seguridad de los trabajadores al reducir la necesidad de intervenciones humanas en entornos subterráneos y potencialmente peligrosos.

	\item El proyecto Tecnobosque \cite{iot_usecase_cuenca} en Cuenca, España, que utiliza drones equipados con sensores e inteligencia artificial para crear cortafuegos preventivos y reducir significativamente las hectáreas de bosques en casos de incendios. 


	\item Spot \cite{spot}, desarrollado por Boston Dynamics, un robot explorador cuadrupedo de propósito general capaz de explorar, almacenar y enviar información en tiempo real.
	  
	\item BIKE \cite{bike_inspection}, desarrollado por Waygate Technologies, un robot con ruedas magnéticas, muy utilizado en la industria de petróleo y gas entre otras, capaz de desplazarse por el interior de tuberías para poder realizar inspecciones y comunicar hallazgos.

	\item El prototipo robótico de exploración minera publicado en varios artículos \cite{latam-mining-robot-minero-unsj}, \cite{diario-de-cuyo-prototipo-robotico}, e impulsado por el Instituto de Automática de la Facultad de Ingeniería de la Universidad Nacional de San Juan en el marco de un convenio con la Comisión Nacional de Energía Atómica y el Gobierno argentino \cite{comunicacion-unsj-prototipo-convenio}.

	\item El robot de exploración terrestre denominado Geobot \cite{geobot} desarrollado por los ingenieros Nelson Dario García Hurtado y Melvin Andrés González Pino, de la universidad de Pamplona, capaz de realizar reconocimiento de zonas y manipulación de muestras de manera autónoma o asistida.

	\item El robot minero MIN-SIS 1.0 SDG-STR \cite{min-sis} desarrollado por los ingenieros Hernán L. Helguero Velásquez y Rubén Medinaceli Tórrez de la Universidad Técnica de Oruro, capaz de detectar gases, almacenar datos locales y enviar video e imágenes al puesto de mando.


\end{itemize}


%----------------------------------------------------------------------------------------

\section{Motivación del trabajo}


La motivación del presente trabajo fue primeramente volcar y unificar en un emprendimiento personal los conceptos aprendidos en la maestría de Internet de las Cosas. 

Se pretendió implementar una arquitectura robusta y flexible que pueda ser extrapolada a casos de uso en la industria en los que sea necesaria la integración de un sistema embebido con una red blockchain para el almacenamiento transparente e inmutable de datos sensibles.  

Por otra parte, se buscó desarrollar un producto cuya implementación pueda contribuir a aumentar el conocimiento público y el estado del arte de proyectos de código abierto para soluciones IoT integradas a blockchain en Argentina.

%----------------------------------------------------------------------------------------

\section{Alcance y objetivos}

A continuación se detallan las funcionalidades incluidas en el alcance del trabajo.


\begin{itemize}
	\item La publicación del endpoint MQTT para la recepción de los datos enviados por el robot.
	\item La adaptación del sistema embebido del robot de exploración ambiental para la conexión segura con el \textit{back-end} vía MQTT.
	\item La arquitectura e implementación de los sistemas \textit{back-end} y el modelo de datos necesario para el almacenamiento de las mediciones enviadas por el robot.
	\item La arquitectura, implementación y despliegue de la dApp y \textit{smart contracts} necesarios para el almacenamiento de las mediciones en una red Blockchain.
	\item La definición de métricas agregadas de valor y posterior arquitectura e implementación de los sistemas analíticos para procesar de forma \textit{batch} y/o \textit{real-time} utilizando herramientas de procesamiento paralelo basadas en big data.
	\item La implementación de la interfaz gráfica para poder visualizar los datos enviados y analíticas calculadas.

\end{itemize}



%----------------------------------------------------------------------------------------

\section{Requerimientos del producto}


A continuación, se listan los requerimientos del producto:

\begin{enumerate}	
	\item Requerimientos funcionales		
	\begin{enumerate}	
		
		\item El robot de exploración ambiental debe poder enviar a la plataforma datos de mediciones de parámetros ambientales, incluyendo los datos de fecha, hora, localización geográfica (que puede ser implementada como un valor \textit{mock}) y la categorización si es o no un valor crítico.
		\item El robot de exploración ambiental debe incorporar lógica para categorizar los valores medidos de cada parámetro ambiental como valores críticos si:
		\begin{enumerate}				
			\item Representan un máximo o mínimo global sensado hasta el momento.				
			\item Representan un máximo o mínimo local durante el último día.				
		\end{enumerate}			
		\item La plataforma debe poder recibir y almacenar las mediciones de parámetros ambientales enviadas por el robot.
		\item Los datos considerados críticos deben ser almacenados en un sistema inmutable.
		\item La plataforma debe poder procesar las mediciones de parámetros ambientales enviadas por el robot para generar métricas de valor para el usuario de negocio.		
		\item La plataforma debe brindar dos \textit{front-end} con interfaz web:
			\begin{enumerate}				
				\item El \textit{front-end} para el usuario de negocio.				
				\item El \textit{front-end} para el usuario administrador.				
			\end{enumerate}			
		
		\item El \textit{front-end} para el usuario de negocio debe proveer métricas para visualizar:
			\begin{enumerate}				
				\item Las lecturas históricas almacenadas.				
				\item Agregaciones (máximo, mínimo, promedio, etc) de cada parámetro ambiental agrupado por frecuencias (ventanas de tiempo) y coordenadas geográficas.				
				\item Las referencias a los datos persistidos en blockchain.
			\end{enumerate}			
		\item El \textit{front-end} para el usuario de administración debe permitir:
			\begin{enumerate}				
				\item Acceder a los diferentes recursos utilizados por la herramienta (topics MQTT, \textit{smart contracts}, \textit{buckets}, etc).
				\item Resetear valores y estado.			
			\end{enumerate}			
		\end{enumerate}	

					
	\item Requerimientos no funcionales		
	\begin{enumerate}	
		\item La plataforma debe contar con al menos un \textit{back-end} de procesamiento y acceso a datos operacionales para la lógica de negocio.
		\item La plataforma debe contar con al menos un \textit{back-end} de acceso, procesamiento, almacenamiento de datos analíticos para la generación de métricas.		
		\item El envío de los valores ambientales censados al \textit{back-end} debe ser mediante MQTT.
		\item Las lecturas ambientales categorizadas como críticas deben ser almacenadas en blockchain para garantizar fiabilidad e inmutabilidad.
		\item La gestión de datos almacenados en blockchain debe ser implementada mediante \textit{smart contracts} desplegados en la red.
		\item La interacción con los \textit{smart contracts} debe realizarse desde una dApp.
		\item Los sistemas de transferencia y almacenamiento de datos utilizados deben contar con seguridad, permitiendo encriptación, autenticación y autorización.	
		\end{enumerate}	
		
	\item Requerimientos de documentación		
		\begin{enumerate}			
			\item Video demostrativo.	
			\item Documentación de arquitectura técnica del diseño del sistema.			
			\item Manual de usuario.	
			\item Memoria final.	
		\end{enumerate}	
		

		
		
	\item Requerimiento de testing		
		\begin{enumerate}			
			\item Se debe incluir tests de unitarios de componentes.
			\item Se debe incluir tests funcionales (\textit{smoke test}) del producto general.		
		\end{enumerate}	
	
	\item Requerimientos opcionales		
		\begin{enumerate}			
			\item De infraestructura y despliegue:
				\begin{enumerate}			
					\item Se permite realizar el despliegue de la dApp en un IPFS (preferentemente) o en la nube.					
					\item Se permite agregado hardware al robot para la captura de datos adicionales.
					\item Se permite agregar automatización para la creación de la infraestructura como código.
				\end{enumerate}			
			
			\item De datos:
				\begin{enumerate}			
					\item Se permite almacenar cualquier otro dato adicional sensado o derivado.
					\item Se permite agregar cualquier implementación de gobierno de datos.	
					\item Se permite almacenar cualquier otra métrica o gráfico de explotación de datos adicional.
				\end{enumerate}
		
	\end{enumerate}
\end{enumerate}